\documentclass[letterpaper, 12pt, dvipsnames]{article}

\usepackage{titling}
\usepackage[margin=2cm]{geometry}

\usepackage[colorlinks]{hyperref}

\usepackage{graphicx}

\usepackage{enumitem}

\usepackage{fontawesome5}

\usepackage{tikz}

\usepackage{changepage} % for adjustwidth environment

\usepackage{svg}

\usepackage{array}

\usepackage{xcolor}

\usepackage{qrcode}

\usepackage[english,spanish]{babel}

\usepackage{titlesec}

% section format

\titleformat*{\section}{\LARGE\bfseries}
\titleformat*{\subsection}{\Large\bfseries}
\titleformat*{\subsubsection}{\large\bfseries}
\titleformat*{\paragraph}{\large\bfseries}
\titleformat*{\subparagraph}{\large\bfseries}

% Setup color links
\hypersetup{
    urlcolor=Blue
}

% Title (author name and rule)

\newcommand{\name}{
    {\Large\bfseries\MakeUppercase{\theauthor}}
}

\renewcommand{\maketitle}{
    \begin{center}
        \name
    \end{center}
}

% Contact info

\newcommand{\emailnoref}{alejandrogomeznoe@gmail.com}
\newcommand{\email}{\href{mailto:\emailnoref}{\emailnoref}}

\newcommand{\linkedinpage}{\href{https://www.linkedin.com/in/alejandro-g\%C3\%B3mez-no\%C3\%A9-bb4239224/}{Alejandro Gómez Noé}}
\newcommand{\githubpage}{\href{https://github.com/algono}{algono}}
\newcommand{\gitlabpage}{\href{https://gitlab.com/algono}{algono}}

% Iconos
\newcommand{\iconoDireccion}{\faIcon{map-marker-alt}}
%\newcommand{\iconoTelefono}{\faPhone}
\newcommand{\iconoTelefono}{\faIcon{phone-alt}}
\newcommand{\iconoEmail}{\faEnvelope}
\newcommand{\iconoNacionalidad}{\faGlobe}
\newcommand{\iconoFechaNacimiento}{\faBirthdayCake}
\newcommand{\iconoDni}{\faIdCard}
\newcommand{\iconoGitHub}{\faGithub}
\newcommand{\iconoGitLab}{\faGitlab}
\newcommand{\iconoLinkedin}{\faLinkedin}

\newcommand{\linkToSelfShort}{bit.ly/3FT3cVY}
\newcommand{\linkToSelf}{https://\linkToSelfShort}

\newcommand{\contactinfo}{
    \begin{minipage}[l]{.25\textwidth}
        \begin{tikzpicture}
        \clip (0,0) circle (2cm) ;
        \node[anchor=center] at (0,0) {\includegraphics[width=4cm,height=4cm,keepaspectratio]{mi-foto.jpg}};
        \end{tikzpicture}
    \end{minipage}
    \begin{minipage}[c]{.45\textwidth}
        \begin{large}
            \iconoEmail \hspace{.5em} \email \par
            \vspace{12pt}
            \iconoDireccion \hspace{.5em} Mislata, Valencia \par
            \vspace{12pt}
            \iconoLinkedin \hspace{.5em} \linkedinpage \par
            \vspace{12pt}
            \iconoGitHub \hspace{.5em} \githubpage \hspace{1em} \iconoGitLab \hspace{.5em} \gitlabpage
        \end{large}
    \end{minipage}
    \begin{minipage}[r]{.25\textwidth}
        \centering
        \href{\linkToSelf}{\linkToSelfShort}
        {
            \hypersetup{
                urlcolor=Black
            }
            \qrcode[padding]{\linkToSelf}
        }
    \end{minipage}
    \begin{center}
    Alejandro Gómez es un \textbf{Ingeniero Informático}\\ especializado en \textbf{Ingeniería del Software},\\actualmente \textbf{trabajando} en la \href{\ujiUrl}{\ujiFull}.
    
    %\\actualmente \textbf{trabajando} en el grupo de investigación \textbf{\href{\sabienUrl}{\itacaSabien}}\\ de la \textbf{\href{\upvUrl}{\upvFull}}.
    \end{center}
}

% Other info
\newcommand{\upvFull}{{\upvName} ({\upv})}
\newcommand{\upvName}{Universitat Politècnica de València}
\newcommand{\upv}{UPV}
\newcommand{\upvUrl}{https://www.upv.es/es}

\newcommand{\sabien}{SABIEN}
\newcommand{\itacaSabien}{ITACA-SABIEN}
\newcommand{\itacaSabienUpv}{{\itacaSabien} (\upv)}
\newcommand{\sabienUrl}{http://www.sabien.upv.es/}

\newcommand{\ujiFull}{{\ujiName} ({\uji})}
\newcommand{\ujiName}{Universitat Jaume I}
\newcommand{\uji}{UJI}
\newcommand{\ujiUrl}{https://www.uji.es/upo/rest/publicacion/idioma/es?urlRedirect=https://www.uji.es/&url=/}
\newcommand{\ujiGroup}{Ingeniería del Conocimiento}
\newcommand{\ujiGroupUrl}{https://www.uji.es/upo/rest/publicacion/idioma/es?urlRedirect=https://www.uji.es/serveis/ocit/base/grupsinvestigacio/detall/&url=/serveis/ocit/base/grupsinvestigacio/detall&codi=158}

\newcommand{\fce}{\emph{First Certificate in English - Grade A - Cambridge English Level 2}}

% My Grades

\newcommand{\grade}[1]{Nota media: #1}
\newcommand{\mh}[1]{Matrículas de honor: #1}

% https://tex.stackexchange.com/questions/150492/how-to-use-itemize-in-table-environment
\newcommand{\tabitem}{~~\llap{\textbullet}~~}

\newcommand{\unigrades}[9]{
    \begin{tabular}{|ll<{\hspace{1em}}|>{\hspace{1em}}ll|}
        \hline
        \multicolumn{4}{|c|}{\textbf{Expediente}}\\[.5em]
        1er curso: & \tabitem \grade{#1} & 3er curso: & \tabitem \grade{#5} \\
        & \tabitem \mh{#2} && \tabitem \mh{#6} \\
        2{\textdegree} curso: & \tabitem \grade{#3} & 4{\textdegree} curso: & \tabitem \grade{#7} \\
        & \tabitem \mh{#4} && \tabitem \mh{#8} \\
        \multicolumn{4}{|c|}{\rule{0pt}{1.5em}Nota final (TFG): #9}\\
        \hline
    \end{tabular}
}

\newcommand{\myunigrades}{
    \unigrades
    {8,3}
    {4}
    {8,1}
    {1}
    {8,3}
    {2}
    {8,4}
    {2}
    {9 (Excelente)}
}

\begin{document}

\title{Curriculum Vitae}
\author{Alejandro Gómez Noé}

\maketitle

\contactinfo

\vspace{8pt}

\rule{.9\textwidth}{.4pt}

\section*{Conocimientos}

\vspace{.8em}

\begin{adjustwidth}{-1.5cm}{}
    \begin{tikzpicture}

        % Define the styles for the circles
        \tikzset{
            skill/.style={
                    align=center,
                    inner sep=0pt,
                    font=\bfseries
                },
            skill-circle/.style={
                    circle,
                    draw,
                    text width=2cm,
                    align=center,
                    inner sep=0pt,
                    font=\bfseries
                },
            category/.style={
                    circle,
                    draw,
                    text width=6cm,
                    minimum height=3cm,
                    align=center,
                    text depth=6cm,
                    font=\bfseries
                },
            category-small/.style={
                    circle,
                    draw,
                    text width=4cm,
                    minimum height=3cm,
                    align=center,
                    text depth=3cm,
                    font=\bfseries,
                },
            category-smaller/.style={
                    circle,
                    draw,
                    text width=3.0583cm,
                    align=center,
                    text depth=3cm,
                    font=\bfseries,
                },
            subcategory/.style={
                    circle,
                    draw,
                    dashed,
                    text width=2.5cm,
                    minimum height=3cm,
                    align=center,
                    text depth=3.5cm,
                    font=\bfseries
                }
        }

        % Draw the categories
        \node[category-smaller] (db) at (1.1,-2.5) {Bases de datos};
        \node[category-smaller, text width=2.5cm, text depth=3cm] (vcs) at (6.8,-0.25) {Control de versiones};
        \node[category-small] (tools) at (5.9,-4.5) {Herramientas};
        \node[subcategory] (dotnet) at (-0.3,2.3) {\includesvg[scale=2]{dotnet.svg}};
        \node[category] (web) at (-5,1) {Desarrollo Web};
        \node[category-smaller, align=center, text depth=1.5cm] (mobile) at (5,3.5) {Móvil};
        \node[category-small, align=right, text depth=0] (desktop) at (2,2) {\colorbox{white}{Escritorio}};

        % Draw the skills within the categories
        \node[skill] (html) at (-8.3,2.5) {{\fontsize{30pt}{30pt}\faHtml5}\\\textbf{HTML}};
        \node[skill] (css) at (-6.5,2.5) {{\fontsize{30pt}{30pt}\faCss3*}\\\textbf{CSS}};
        \node[skill] (js) at (-4.6,2.45) {{\fontsize{30pt}{30pt}\faJs}\\\textbf{JavaScript}};
        \node[skill] (php) at (-6,0.35) {\fontsize{40pt}{40pt}\faPhp};
        \node[skill,text width=2.5cm] (wordpress) at (-8.2,0.25) {{\fontsize{30pt}{30pt}\faWordpress}\\\textbf{WordPress}};
        \node[skill] (elementor) at (-6.5,-1.5) {{\fontsize{30pt}{30pt}\faElementor}\\\textbf{Elementor}};
        \node[skill] (ts) at (-3.8,0.25) {{\fontsize{30pt}{30pt}\includesvg[scale=1.3]{typescript.svg}}\\\textbf{TypeScript}};
        \node[skill] (node) at (-4.3,-2.2) {{\fontsize{30pt}{30pt}\faIcon{node-js}}\\\textbf{Node.js}};
        \node[skill] (docker) at (-2.5,-1.5) {{\fontsize{30pt}{30pt}\faDocker}\\\textbf{Docker}};
        \node[skill, text width=2.1cm] (csharp) at (-0.4,3.1) {{\fontsize{30pt}{30pt}\includesvg[scale=1.3]{csharp.svg}}};
        \node[skill-circle, text width=1.5cm] (wpf) at (1.3,3) {WPF};
        \node[skill] (unity) at (0.9,1.3) {{\fontsize{30pt}{30pt}\faUnity}\\\textbf{Unity}};
        \node[skill, text width=2.1cm] (blazor) at (-1.8,2) {{\fontsize{30pt}{30pt}\includesvg[scale=1.3]{blazor.svg}}\\\textbf{Blazor}};
        \node[skill] (java) at (2.6,5) {{\fontsize{30pt}{30pt}\faJava}\\Java};
        \node[skill] (flutter) at (5.2,3) {{\fontsize{30pt}{30pt}\includesvg[scale=1.1]{flutter.svg}}\\Flutter};
        \node[skill] (sql) at (0,-2.5) {{\fontsize{30pt}{30pt}\faDatabase}\\SQL};
        \node[skill] (firebase) at (1.9,-2.5) {{\fontsize{30pt}{30pt}\includesvg[scale=1.2]{firebase.svg}}\\Firebase};
        \node[skill] (alexa) at (-2,-3.5) {{\fontsize{30pt}{30pt}\includesvg[scale=1.2]{alexa.svg}}\\Alexa};
        \node[skill] (linux) at (3.9,-0.6) {{\fontsize{30pt}{30pt}\faLinux}\\Linux};
        \node[skill] (git) at (5.7,-0.5) {{\fontsize{30pt}{30pt}\faGit}};
        \node[skill] (github) at (7.5,-0.5) {{\fontsize{30pt}{30pt}\faGithub}\\GitHub};
        \node[skill] (vscode) at (4.6,-4.2) {{\fontsize{30pt}{30pt}\includesvg[scale=1.1]{visual-studio.svg}}\\Visual Studio};
        \node[skill] (vscode) at (7.3,-4.2) {{\fontsize{30pt}{30pt}\includesvg[scale=1.1]{vscode.svg}}\\VS Code};
        \node[skill] (trello) at (6,-6) {{\fontsize{30pt}{30pt}\faTrello}\\Trello};
        \node[skill] (latex) at (2.25,-5.5) {\Large\LaTeX};
    \end{tikzpicture}
\end{adjustwidth}

\pagebreak

\section*{Experiencia}

\subsection*{Personal de apoyo a la investigación | \emph{\ujiFull}\\\textit{\large (mayo 2024 - febrero 2025)}}

Como \textbf{personal de apoyo a la investigación} en el grupo \emph{\textbf{\href{\ujiGroupUrl}{\ujiGroup}}} de la \emph{\href{\ujiUrl}{\ujiFull}}, participé en el \textbf{proyecto \emph{MINEGUIDE}}, cuyo objetivo es el desarrollo de un sistema de \textbf{apoyo a la toma de decisiones} en el ámbito de la \textbf{salud} mediante \textbf{minería de procesos} y \textbf{guías clínicas}. Se trata de una colaboración entre la \emph{\ujiFull}, la \emph{\upvFull} y la \emph{Universidad de Murcia (UMU)}.

\subsubsection*{Proyecto MINEGUIDE}

Mi principal contribución fue realizar un \textbf{análisis de caso} aplicando técnicas de \textbf{minería de procesos} a datos hospitalarios relacionados con la \textbf{EPOC} (\emph{Enfermedad Pulmonar Obstructiva Crónica}), identificando \textbf{relaciones} y \textbf{desviaciones} entre los \textbf{procesos reales} (obtenidos de los datos) y los \textbf{procesos ideales} (definidos en las guías clínicas), con el objetivo de publicar los resultados en un artículo científico.\\

Durante el proceso, también realicé mejoras en la \textbf{herramienta} de \textbf{minería de procesos} utilizada, \textbf{PMApp} (desarrollada por el grupo \emph{\sabien} de la \emph{\upv}), además de utilizar \textbf{\emph{LLMs}} para analizar y extraer información de datos médicos en texto libre, automatizando el proceso y desplegando uno propio para incrementar la seguridad.

\subsection*{Técnico medio en informática | \emph{\itacaSabienUpv}\\\textit{\large (marzo 2022 - abril 2024)}}

Como \textbf{técnico medio en informática} en el grupo \emph{\textbf{\href{\sabienUrl}{\sabien}} (Innovaciones Tecnológicas para la Salud y el Bienestar)} del \emph{Instituto de Aplicaciones de las Tecnologías de la Información y de las Comunicaciones Avanzadas (\textbf{\href{http://www.itaca.upv.es/}{ITACA}})} de la \emph{{\upvName} (\textbf{\href{\upvUrl}{\upv}})}, mi trabajo consiste en el \textbf{desarrollo de aplicaciones} y el \textbf{apoyo informático} de distinta índole en el marco de diversos \textbf{proyectos de investigación}.

\subsubsection*{Proyectos en los que he participado}

\vspace{1em}

\begin{itemize}
    \item {\large\textbf{\href{https://moveit.webs.upv.es/es/}{MOVE-IT}}} (2023 - 2024)
          \begin{itemize}
              \item Programa de entrenamiento para mejorar el ejercicio físico de las personas con discapacidad intelectual mediante \textit{exergames} y tecnología
              \item \textbf{Proyecto europeo ERASMUS+} en colaboración con:
                    \begin{itemize}
                        \item \textbf{\href{https://www.ivass.gva.es/}{IVASS}} - Instituto Valenciano de Servicios Sociales (España)
                        \item \textbf{\href{https://www.cercioeiras.pt/pt}{CERCIOEIRAS}} - \textit{Cooperativa de Educação e Reabilitação de Cidadãos com Incapacidade} (Portugal)
                        \item \textbf{\href{https://en.uit.no/}{UiT}} - Universidad de Tromsø (Noruega)
                        \item \textbf{\href{https://www.ospedalemotta.it/it/}{ORAS}} - \textit{Ospedale riabilitativo di Motta di Livenza} - Hospital (Italia)
                    \end{itemize}
            \item Mejoré y refiné las aplicaciones desarrolladas por \textbf{UiT} con \textbf{Unity} (\textit{AGA} y \textit{Sorterius}), abordando problemas existentes y optimizando la funcionalidad
            \item Implementé un backend personalizado para el proyecto, incluyendo un sitio web con una API para la recopilación de datos utilizando \textbf{Blazor}, \textbf{ASP.NET Core Identity} y \textbf{EFCore}, que luego alojé en un servidor de la \textit{\upv}
            \item Integré con éxito la aplicación \textit{Sorterius} con la API del backend para la gestión de usuarios y la recopilación de datos durante los pilotos para su posterior estudio
            \item Gestioné el servidor y trabajé junto a los centros participantes durante los pilotos para asegurar el correcto funcionamiento de las aplicaciones
            \item Las pruebas piloto involucraron a un total de \textbf{17 participantes} durante un periodo de \textbf{2 semanas}, con un total de \textbf{158 sesiones} de juego. Un \textbf{77\%} de los participantes \textbf{aumentaron su uso de \emph{smartphones}} como resultado del uso de la aplicación, y su motivación permaneció alta durante todo el periodo de prueba
            \item Los resultados se publicaron en el siguiente \textbf{artículo científico} del que soy co-autor: <<A. Henriksen, A. Martinez-Millana, A. Gomez-Noe, G. Hartvigsen, A. Anke, M. Stellander, D. Dybwad, T. Luzi, H. Michalsen, Piloting an Augmented Reality Exergame for Persons with Intellectual Disabilities, Intelligent Health Systems – From Technology to Data and Knowledge 2025. \url{https://doi.org/10.3233/shti250600}>>.
          \end{itemize}
    \item {\large\textbf{\href{https://pm4health.com/}{PM4H}}} (2022 - 2024)
          \begin{itemize}
              \item He desarrollado varias funcionalidades para la \textbf{aplicación de escritorio} <<\textbf{\emph{PMApp}}>> (desarrollada en \textbf{C\#} con \textbf{WPF}), la cual ejecuta algoritmos de \textbf{\emph{process mining}} para tratar y visualizar diferentes datos
              \item He mejorado el sistema de visualización de tablas e histogramas, he añadido soporte para el uso de \emph{proxies}\dots entre otras.
          \end{itemize}
    \item {\large\textbf{\href{https://lifechamps.eu/}{LIFECHAMPS}}} (2023)
        \begin{itemize}
            \item \textbf{Proyecto europeo} para la mejora de la calidad de vida de pacientes con cáncer, formado por un consorcio de 15 socios de 10 países, liderado por la \href{https://www.auth.gr/en/}{Aristotle University of Thessaloniki}
            \item Colaboré estrechamente con el \textbf{\href{https://www.iislafe.es/es/}{IIS La Fe}} y la empresa \textbf{\href{https://www.mysphera.com/es/}{MySphera}}, \textbf{configurando} dispositivos \textbf{Raspberry Pi} y realizando un total de \textbf{50 instalaciones} en casas de pacientes para un proyecto piloto a nivel multinacional, colocando a España como uno de los países con más instalaciones en el proyecto
            \item Además, ofrecí soporte técnico a los pacientes y a los investigadores de La Fe
        \end{itemize}
    \item {\large\textbf{\href{http://www.sabien.upv.es/project/dial/}{DIAL}}} (2022 - 2023)
          \begin{itemize}
              \item \textbf{Asistente de voz} para la detección y abordaje de Soledad No Deseada en personas mayores, basado en el sistema de código abierto (\emph{open source}) \textbf{\href{https://mycroft.ai/}{Mycroft}}
              \item Entrené y probé un modelo de \textbf{\emph{machine learning}} para poder activar el asistente con la frase <<\emph{Hola dial}>>
              \item Además, \textbf{configuré} y probé más de 20 dispositivos \textbf{Raspberry Pi} para que funcionen como asistentes de voz mediante el sistema de DIAL
          \end{itemize}
    \item {\large\textbf{\href{https://orriolsarrandeterra.com/}{Orriols Arran de Terra}}} (2022)
          \begin{itemize}
              \item \textbf{Página web} destinada a la difusión de noticias y actividades del barrio de \emph{Els Orriols} (Valencia)
              \item Desarrollé la web con \textbf{WordPress} y \textbf{Elementor}
              \item Para implementar algunas funcionalidades personalizadas, creé \textbf{plugins} con \textbf{PHP}
              \item Además utilicé \textbf{HTML}, \textbf{CSS} y \textbf{JavaScript}
          \end{itemize}
\end{itemize}

\section*{Educación}

\subsection*{Universidad}

\fbox{
    \large
    \begin{minipage}{.6\textwidth}
        Grado en Ingeniería Informática

        Mención en Ingeniería del Software

            {\upvFull}, 2021
    \end{minipage}
}

\subsubsection*{Proyectos universitarios}

\begin{itemize}
    \item \textbf{Al Loro} (\href{https://github.com/algono/FeedTheParrot-RSS}{Repositorio}) (\href{http://hdl.handle.net/10251/174256}{Memoria}):
          \begin{itemize}
              \item Implementé en solitario una \textbf{skill} para Amazon \textbf{Alexa} como Trabajo de Fin de Grado (\textbf{TFG}), usando \textbf{Node.js} y \textbf{TypeScript}
              \item Integré la skill con una \textbf{base de datos} en \textbf{Firebase}, creando además una app con \textbf{Flutter} para gestionar las preferencias del usuario
              \item Diseñé un sistema de autenticación usando servicios de \textbf{AWS} como \textbf{Lambda}, \textbf{DynamoDB} o \textbf{API Gateway}
          \end{itemize}
    % \item \textbf{Frozen Out} (\href{https://github.com/Freezer-Games/Frozen-Out}{Repositorio}):
    %       \begin{itemize}
    %           \item Participé en el desarrollo de un videojuego hecho en \textbf{Unity} con \textbf{C\#} como proyecto para la asignatura \emph{IPV} (2019)
    %           \item Diseñé un \textbf{sistema de diálogos} con soporte para formatos como negrita, cursiva, y distintos colores apoyándome en la librería \href{https://yarnspinner.dev/}{\textbf{\emph{YarnSpinner}}}
    %           \item El proyecto participó en la \href{https://es-es.facebook.com/etsinf/videos/feria-de-proyectos-de-estudiantes-2019/1921312964681641/}{\textbf{Feria de proyectos} (2019)} que organizó la \emph{ETSINF}
    %           \item El proyecto continuó sin mí tras la entrega final de la asignatura (Enero de 2020). En Febrero de 2021, \emph{Frozen Out} recibió el \textbf{Premio Especial Compromiso PlayStation} (\href{https://www.inf.upv.es/www/etsinf/es/premio-especial-compromiso-playstation-para-el-videojuego-frozen-out-creado-por-estudiantes-de-la-etsinf-y-la-facultat-de-bb-aa/}{artículo})
    %       \end{itemize}
\end{itemize}

\section*{Idiomas}

\begin{itemize}
    \item \textbf{Español}, nativo
    \item \textbf{Inglés}, nivel \textbf{C1} (\fce)
    \item Valenciano, nivel C1
\end{itemize}

\section*{Actividades}

\subsection*{Mentor - Technovation Challenge}

Participé como mentor voluntario en el concurso de Iridescent \href{https://technovationchallenge.org/}{Technovation Challenge} en su edición del año 2019, en colaboración con \href{https://americanspacev.upv.es/}{American Space}, una asociación de la \emph{\upvFull}.

% \section*{Otros}

% \begin{itemize}
%     \item Carnet de conducir (clase B)
% \end{itemize}

\end{document}