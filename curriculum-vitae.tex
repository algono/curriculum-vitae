\documentclass[letterpaper, 12pt, dvipsnames]{article}

\usepackage{titling}
\usepackage[margin=2cm]{geometry}

\usepackage[colorlinks]{hyperref}

\usepackage{graphicx}

\usepackage{enumitem}

\usepackage{fontawesome5}

\usepackage{tikz}

\usepackage{changepage} % for adjustwidth environment

\usepackage{svg}

\usepackage{array}

\usepackage{xcolor}

\usepackage{qrcode}

\usepackage[english,spanish]{babel}

% Setup color links
\hypersetup{
    urlcolor=Blue
}

% Title (author name and rule)

\newcommand{\name}{
    {\Large\bfseries\MakeUppercase{\theauthor}}
}

\renewcommand{\maketitle}{
    \begin{center}
        \name
    \end{center}
}

% Contact info

\newcommand{\emailnoref}{alejandrogomeznoe@gmail.com}
\newcommand{\email}{\href{mailto:\emailnoref}{\emailnoref}}

\newcommand{\linkedinpage}{\href{https://www.linkedin.com/in/alejandro-g\%C3\%B3mez-no\%C3\%A9-bb4239224/}{Alejandro Gómez Noé}}
\newcommand{\githubpage}{\href{https://github.com/algono}{algono}}
\newcommand{\gitlabpage}{\href{https://gitlab.com/algono}{algono}}

% Iconos
\newcommand{\iconoDireccion}{\faIcon{map-marker-alt}}
%\newcommand{\iconoTelefono}{\faPhone}
\newcommand{\iconoTelefono}{\faIcon{phone-alt}}
\newcommand{\iconoEmail}{\faEnvelope}
\newcommand{\iconoNacionalidad}{\faGlobe}
\newcommand{\iconoFechaNacimiento}{\faBirthdayCake}
\newcommand{\iconoDni}{\faIdCard}
\newcommand{\iconoGitHub}{\faGithub}
\newcommand{\iconoGitLab}{\faGitlab}
\newcommand{\iconoLinkedin}{\faLinkedin}

\newcommand{\linkToSelfShort}{bit.ly/3FT3cVY}
\newcommand{\linkToSelf}{https://\linkToSelfShort}

\newcommand{\contactinfo}{
    \begin{minipage}[l]{.25\textwidth}
        \begin{tikzpicture}
        \clip (0,0) circle (2cm) ;
        \node[anchor=center] at (0,0) {\includegraphics[width=4cm,height=4cm,keepaspectratio]{mi-foto.jpg}};
        \end{tikzpicture}
    \end{minipage}
    \begin{minipage}[c]{.45\textwidth}
        \begin{large}
            \iconoEmail \hspace{.5em} \email \par
            \vspace{12pt}
            \iconoDireccion \hspace{.5em} Mislata, Valencia \par
            \vspace{12pt}
            \iconoLinkedin \hspace{.5em} \linkedinpage \par
            \vspace{12pt}
            \iconoGitHub \hspace{.5em} \githubpage \hspace{1em} \iconoGitLab \hspace{.5em} \gitlabpage
        \end{large}
    \end{minipage}
    \begin{minipage}[r]{.25\textwidth}
        \centering
        \href{\linkToSelf}{\linkToSelfShort}
        {
            \hypersetup{
                urlcolor=Black
            }
            \qrcode[padding]{\linkToSelf}
        }
    \end{minipage}
    \begin{center}
    Alejandro Gómez es un \textbf{Ingeniero Informático}\\ especializado en \textbf{Ingeniería del Software},\\actualmente \textbf{trabajando} en el grupo de investigación \textbf{ITACA-SABIEN}\\ de la \textbf{\uni}.
    \end{center}
}

% Other info
\newcommand{\uni}{Universitat Politècnica de València (UPV)}

\newcommand{\fce}{\emph{First Certificate in English - Grade A - Cambridge English Level 2}}

% My Grades

\newcommand{\grade}[1]{Nota media: #1}
\newcommand{\mh}[1]{Matrículas de honor: #1}

% https://tex.stackexchange.com/questions/150492/how-to-use-itemize-in-table-environment
\newcommand{\tabitem}{~~\llap{\textbullet}~~}

\newcommand{\unigrades}[9]{
    \begin{tabular}{|ll<{\hspace{1em}}|>{\hspace{1em}}ll|}
        \hline
        \multicolumn{4}{|c|}{\textbf{Expediente}}\\[.5em]
        1er curso: & \tabitem \grade{#1} & 3er curso: & \tabitem \grade{#5} \\
        & \tabitem \mh{#2} && \tabitem \mh{#6} \\
        2{\textdegree} curso: & \tabitem \grade{#3} & 4{\textdegree} curso: & \tabitem \grade{#7} \\
        & \tabitem \mh{#4} && \tabitem \mh{#8} \\
        \multicolumn{4}{|c|}{\rule{0pt}{1.5em}Nota final (TFG): #9}\\
        \hline
    \end{tabular}
}

\newcommand{\myunigrades}{
    \unigrades
    {8,3}
    {4}
    {8,1}
    {1}
    {8,3}
    {2}
    {8,4}
    {2}
    {9 (Excelente)}
}

\begin{document}

\title{Curriculum Vitae}
\author{Alejandro Gómez Noé}

\maketitle

\contactinfo

\vspace{8pt}

\rule{.9\textwidth}{.4pt}

\section{Conocimientos}

\vspace{.8em}

\begin{adjustwidth}{-1.5cm}{}
    \begin{tikzpicture}

        % Define the styles for the circles
        \tikzset{
            skill/.style={
                    align=center,
                    inner sep=0pt,
                    font=\bfseries
                },
            skill-circle/.style={
                    circle,
                    draw,
                    text width=2cm,
                    align=center,
                    inner sep=0pt,
                    font=\bfseries
                },
            category/.style={
                    circle,
                    draw,
                    text width=6cm,
                    minimum height=3cm,
                    align=center,
                    text depth=6cm,
                    font=\bfseries
                },
            category-small/.style={
                    circle,
                    draw,
                    text width=4cm,
                    minimum height=3cm,
                    align=center,
                    text depth=3cm,
                    font=\bfseries,
                },
            category-smaller/.style={
                    circle,
                    draw,
                    text width=3.0583cm,
                    align=center,
                    text depth=3cm,
                    font=\bfseries,
                },
            subcategory/.style={
                    circle,
                    draw,
                    dashed,
                    text width=2.5cm,
                    minimum height=3cm,
                    align=center,
                    text depth=3.5cm,
                    font=\bfseries
                }
        }

        % Draw the categories
        \node[category-smaller] (db) at (1.1,-2.5) {Bases de datos};
        \node[category-smaller, text width=2.5cm, text depth=3cm] (vcs) at (6.8,-0.25) {Control de versiones};
        \node[category-small] (tools) at (5.9,-4.5) {Herramientas};
        \node[subcategory] (dotnet) at (-0.5,2.5) {\includesvg[scale=2]{csharp.svg}};
        \node[category] (web) at (-5,1) {Desarrollo Web};
        \node[category-smaller, align=center, text depth=1.5cm] (mobile) at (5,3.5) {Móvil};
        \node[category-small, align=right, text depth=0] (desktop) at (2,2) {\colorbox{white}{Escritorio}};

        % Draw the skills within the categories
        \node[skill] (html) at (-8.3,2.5) {{\fontsize{30pt}{30pt}\faHtml5}\\\textbf{HTML}};
        \node[skill] (css) at (-6.5,2.5) {{\fontsize{30pt}{30pt}\faCss3*}\\\textbf{CSS}};
        \node[skill] (js) at (-4.6,2.45) {{\fontsize{30pt}{30pt}\faJs}\\\textbf{JavaScript}};
        \node[skill] (php) at (-6,0.35) {\fontsize{40pt}{40pt}\faPhp};
        \node[skill,text width=2.5cm] (wordpress) at (-8.2,0.25) {{\fontsize{30pt}{30pt}\faWordpress}\\\textbf{WordPress}};
        \node[skill] (elementor) at (-6.5,-1.5) {{\fontsize{30pt}{30pt}\faElementor}\\\textbf{Elementor}};
        \node[skill] (ts) at (-3.8,0.25) {{\fontsize{30pt}{30pt}\includesvg[scale=1.3]{typescript.svg}}\\\textbf{TypeScript}};
        \node[skill] (node) at (-4.3,-2.2) {{\fontsize{30pt}{30pt}\faIcon{node-js}}\\\textbf{Node.js}};
        \node[skill] (docker) at (-2.5,-1.5) {{\fontsize{30pt}{30pt}\faDocker}\\\textbf{Docker}};
        \node[skill-circle, text width=1.5cm] (wpf) at (1.3,3) {WPF};
        \node[skill] (unity) at (0.9,1.3) {{\fontsize{30pt}{30pt}\faUnity}\\\textbf{Unity}};
        \node[skill-circle, text width=2.1cm] (aspnetcore) at (-2,2) {ASP.NET Core};
        \node[skill] (java) at (2.6,5) {{\fontsize{30pt}{30pt}\faCoffee}\\Java};
        \node[skill] (flutter) at (5.2,3) {{\fontsize{30pt}{30pt}\includesvg[scale=1.1]{flutter.svg}}\\Flutter};
        \node[skill] (sql) at (0,-2.5) {{\fontsize{30pt}{30pt}\faDatabase}\\SQL};
        \node[skill] (firebase) at (1.9,-2.5) {{\fontsize{30pt}{30pt}\includesvg[scale=1.2]{firebase.svg}}\\Firebase};
        \node[skill] (alexa) at (-2,-3.5) {{\fontsize{30pt}{30pt}\includesvg[scale=1.2]{alexa.svg}}\\Alexa};
        \node[skill] (linux) at (3.9,-0.6) {{\fontsize{30pt}{30pt}\faLinux}\\Linux};
        \node[skill] (git) at (5.7,-0.5) {{\fontsize{30pt}{30pt}\faGit}};
        \node[skill] (github) at (7.5,-0.5) {{\fontsize{30pt}{30pt}\faGithub}\\GitHub};
        \node[skill] (vscode) at (4.6,-4.2) {{\fontsize{30pt}{30pt}\includesvg[scale=1.1]{visual-studio.svg}}\\Visual Studio};
        \node[skill] (vscode) at (7.3,-4.2) {{\fontsize{30pt}{30pt}\includesvg[scale=1.1]{vscode.svg}}\\VS Code};
        \node[skill] (trello) at (4.8,-6) {{\fontsize{30pt}{30pt}\faTrello}\\Trello};
        \node[skill] (vscode) at (6.8,-6) {{\fontsize{30pt}{30pt}\includesvg{sonarqube.svg}}\\\rotatebox{10}{Sonarqube}};
    \end{tikzpicture}
\end{adjustwidth}

%\renewcommand{\arraystretch}{1.5}
%\begin{LARGE}
%    \begin{tabular}{ll<{\hspace{1em}}|>{\hspace{1em}}l}
%        \multicolumn{2}{c}{\Large{\textbf{\textcolor{OliveGreen}{Lenguajes}}}}      & \Large{\textbf{\textcolor{RawSienna}{Tecnologías}}}                                                                 \\[.5ex]
%        \raisebox{-.12em}{\includesvg[scale=1.3]{csharp.svg}} C\#                   & \textcolor[HTML]{A5630B}{\faCoffee} \hspace{.1em} Java   & \textcolor[HTML]{339933}{\faIcon{node-js}} Node.js       \\
%        \includesvg[scale=1.2]{typescript.svg} TypeScript                           & \textcolor[HTML]{F7DF1E}{\faJs} \hspace{.1em} JavaScript & \textcolor[HTML]{F05032}{\faIcon{git-alt}} Git           \\
%        \includesvg[scale=1.15]{dart.svg} Dart                                      & \faDatabase \hspace{.1em} SQL                            & \iconoGitHub \hspace{.1em} GitHub                        \\
%        \faPython \hspace{.1em} Python                                              & \faLinux \hspace{.1em} Bash                              & \includesvg[scale=1.2]{firebase.svg} Firebase            \\
%        \faIcon{php} PHP                                                            & \LaTeX                                                   & \includesvg[scale=1.1]{flutter.svg} Flutter              \\
%                                                                                    &                                                          & \faUnity \hspace{.1pt} Unity                             \\\cline{1-2}
%        \multicolumn{2}{c|}{\Large{\textbf{\textcolor{RoyalPurple}{Herramientas}}}} & \textcolor[HTML]{0078D6}{\faWindows} \hspace{.1em} WPF                                                              \\
%        \includesvg[scale=1.1]{vscode.svg} VSCode                                   & \textcolor[HTML]{0052CC}{\faTrello} \hspace{.1em} Trello & \textcolor[HTML]{2496ED}{\faDocker} \hspace{.1em} Docker \\
%        \includesvg[scale=1.1]{visual-studio.svg} Visual Studio                     & \includesvg{sonarqube.svg} Sonarqube                     & \includesvg[scale=1.2]{alexa.svg} Alexa                  \\
%    \end{tabular}
%\end{LARGE}

\pagebreak

\section{Experiencia}

\begin{center}
    \large{\textbf{Al Loro - Desarrollador principal}}\par
    \vspace{.2em}
    \href{https://github.com/algono/FeedTheParrot-RSS}{Repositorio}\hspace{1em}\href{http://hdl.handle.net/10251/174256}{Memoria}
\end{center}
\begin{itemize}
    \item Implementé en solitario una \textbf{skill} para Amazon \textbf{Alexa} como Trabajo de Fin de Grado (\textbf{TFG}), usando \textbf{Node.js} y \textbf{TypeScript}.
    \item Integré la skill con una \textbf{base de datos} en \textbf{Firebase}, creando además una app con \textbf{Flutter} para gestionar las preferencias del usuario.
    \item Diseñé un sistema de autenticación usando servicios de \textbf{AWS} como \textbf{Lambda}, \textbf{DynamoDB} o \textbf{API Gateway}.
\end{itemize}
\rule{\textwidth}{.4pt}

\vspace{.5em}

Los siguientes proyectos participaron en la \href{https://es-es.facebook.com/etsinf/videos/feria-de-proyectos-de-estudiantes-2019/1921312964681641/}{\textbf{Feria de proyectos} (2019)} que organizó la \emph{ETSINF}:

\vspace{.3em}

\begin{center}
    \large{\textbf{Transportify - Desarrollador Jefe}}\par
    \vspace{.2em}
    \href{https://github.com/Mobility-Solutions/Transportify}{Repositorio}
\end{center}
\begin{itemize}
    \item Lideré un equipo de 8 personas en la creación de una app con \textbf{Flutter} para la asignatura \emph{PIN}.
    \item Desarrollé una serie de componentes para facilitar la integración de la app con la base de datos, diseñada en \textbf{Firebase}.
    \item Gestioné el proyecto mediante \textbf{metodologías ágiles} a través de la plataforma \href{http://www.tuneupprocess.com/}{\textbf{\emph{Worki}}}, desarrollada por nuestro profesor; además del sistema de control de versiones en \textbf{GitHub}, tratando de seguir un flujo de trabajo basado en \textbf{Git flow}.
\end{itemize}
\vspace{1em}
\begin{center}
    \large{\textbf{Frozen Out - Desarrollador}}\par
    \vspace{.2em}
    \href{https://github.com/Mobility-Solutions/Transportify}{Repositorio}
\end{center}
\begin{itemize}
    \item Participé en el desarrollo de un videojuego hecho en \textbf{Unity} con \textbf{C\#} como proyecto para la asignatura \emph{IPV}.
    \item Diseñé un \textbf{sistema de diálogos} con soporte para formatos como negrita, cursiva, y distintos colores apoyándome en la librería \href{https://yarnspinner.dev/}{\textbf{\emph{YarnSpinner}}}.
    \item El proyecto continuó sin mí tras la entrega final de la asignatura (Enero de 2020). En Febrero de 2021, \emph{Frozen Out} recibió el \textbf{Premio Especial Compromiso PlayStation} (\href{https://www.inf.upv.es/www/etsinf/es/premio-especial-compromiso-playstation-para-el-videojuego-frozen-out-creado-por-estudiantes-de-la-etsinf-y-la-facultat-de-bb-aa/}{artículo}).
\end{itemize}
\rule{\textwidth}{.4pt}

\pagebreak

\begin{center}
    \large{\textbf{CLIPS IDE - Desarrollador principal}}\par
    \vspace{.2em}
    \href{https://marketplace.visualstudio.com/items?itemName=algono.clips-ide}{Página web}
    \hspace{1em}\href{https://github.com/algono/clips-ide-vscode}{Repositorio}
\end{center}
\begin{itemize}
    \item Diseñé por cuenta propia una \textbf{extensión} para el editor de código \textbf{Visual Studio Code}, utilizando \textbf{Node.js} y \textbf{TypeScript}.
    \item Este complemento añade \textbf{soporte} para el lenguaje de programación \textbf{CLIPS}; incluyendo un \textbf{\emph{REPL}} (un "<entorno de programación interactiva">\footnote{\url{https://es.wikipedia.org/wiki/REPL}} basado en un terminal), además de otras ventanas con información útil para programar en dicho lenguaje.

          Esto aporta una experiencia similar a un \textbf{\emph{IDE}} (entorno de desarrollo integrado).

          \includegraphics[width=.9415\textwidth]{vscode-clips-ide.png}

    \item La idea surgió gracias a que me enseñaron este lenguaje en la asignatura \emph{SIN} de 3er año en el grado.

          Intentando encontrar aplicaciones que me habría gustado tener mientras estudiaba, recordé el lenguaje \emph{CLIPS}, y cómo me habría gustado que tuviera soporte para mi principal editor de código.
    \item En los \textbf{primeros 4 meses}, la extensión ha sido \textbf{instalada más de 600 veces}, y ha recibido una \textbf{valoración de 5 estrellas} (sobre 5).
\end{itemize}

\subsection{Otros proyectos}

\begin{itemize}
    \item \textbf{London Eye} (\href{https://github.com/algono/London-Eye}{repositorio}): Videojuego hecho en \textbf{Unity} con \textbf{C\#} en 3 semanas para la asignatura \emph{EDV}. En este juego mejoré el sistema de diálogos que hice para \emph{Frozen Out}.
    \item \textbf{Acceso UPV} (\href{https://github.com/algono/AccesoUPV}{repositorio}): Aplicación de escritorio para Windows, diseñada para acceder fácilmente a varios servicios de la UPV (VPN, Disco W, etc). Hecha en \textbf{C\#} con \textbf{WPF}.
\end{itemize}

\section{Educación}

\subsection{Universidad}

{\uni}

Grado en Ingeniería Informática, 2021

Mención en Ingeniería del Software

\subsection{Campus Científicos - 2015}

Durante el verano de 2015, participé en los \href{https://www.campuscientificos.es/}{Campus Científicos} (organizados por la FECYT y el ministerio de Educación), cursando el proyecto de ``Seguridad en Redes e Internet'' en la universidad Carlos III de Madrid.

\section{Idiomas}

\begin{itemize}
    \item \textbf{Español}, nativo
    \item \textbf{Inglés}, nivel \textbf{C1} (\fce)
    \item Valenciano, nivel B1
\end{itemize}

\section{Actividades}

\subsection{Mentor - Technovation Challenge}

Participé como mentor voluntario en el concurso de Iridescent \href{https://technovationchallenge.org/}{Technovation Challenge} en su edición del año 2019, en colaboración con el \href{https://cdl.upv.es/american-space}{American Space}, una asociación de la \emph{\uni}.

\section{Otros}

\begin{itemize}
    \item Carnet de conducir (tipo B).
\end{itemize}

\end{document}