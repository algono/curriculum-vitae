\documentclass[letterpaper, 12pt, dvipsnames]{article}

\usepackage{titling}
\usepackage[margin=2cm]{geometry}

\usepackage[colorlinks]{hyperref}

\usepackage{graphicx}

\usepackage{enumitem}

\usepackage{fontawesome5}

\usepackage{tikz}

\usepackage{changepage} % for adjustwidth environment

\usepackage{svg}

\usepackage{array}

\usepackage{xcolor}

\usepackage{qrcode}

\usepackage[spanish,english]{babel}

\usepackage{titlesec}

% section format

\titleformat*{\section}{\LARGE\bfseries}
\titleformat*{\subsection}{\Large\bfseries}
\titleformat*{\subsubsection}{\large\bfseries}
\titleformat*{\paragraph}{\large\bfseries}
\titleformat*{\subparagraph}{\large\bfseries}

% Setup color links
\hypersetup{
    urlcolor=Blue
}

% Title (author name and rule)

\newcommand{\name}{
    {\Large\bfseries\MakeUppercase{\theauthor}}
}

\renewcommand{\maketitle}{
    \begin{center}
        \name
    \end{center}
}

% Contact info

\newcommand{\emailnoref}{alejandrogomeznoe@gmail.com}
\newcommand{\email}{\href{mailto:\emailnoref}{\emailnoref}}

\newcommand{\linkedinpage}{\href{https://www.linkedin.com/in/alejandro-g\%C3\%B3mez-no\%C3\%A9-bb4239224/}{Alejandro Gómez Noé}}
\newcommand{\githubpage}{\href{https://github.com/algono}{algono}}
\newcommand{\gitlabpage}{\href{https://gitlab.com/algono}{algono}}

% Iconos
\newcommand{\iconoDireccion}{\faIcon{map-marker-alt}}
%\newcommand{\iconoTelefono}{\faPhone}
\newcommand{\iconoTelefono}{\faIcon{phone-alt}}
\newcommand{\iconoEmail}{\faEnvelope}
\newcommand{\iconoNacionalidad}{\faGlobe}
\newcommand{\iconoFechaNacimiento}{\faBirthdayCake}
\newcommand{\iconoDni}{\faIdCard}
\newcommand{\iconoGitHub}{\faGithub}
\newcommand{\iconoGitLab}{\faGitlab}
\newcommand{\iconoLinkedin}{\faLinkedin}

\newcommand{\linkToSelfShort}{bit.ly/463OXvZ}
\newcommand{\linkToSelf}{https://\linkToSelfShort}

\newcommand{\contactinfo}{
    \begin{minipage}[l]{.25\textwidth}
        \begin{tikzpicture}
        \clip (0,0) circle (2cm) ;
        \node[anchor=center] at (0,0) {\includegraphics[width=4cm,height=4cm,keepaspectratio]{mi-foto.jpg}};
        \end{tikzpicture}
    \end{minipage}
    \begin{minipage}[c]{.45\textwidth}
        \begin{large}
            \iconoEmail \hspace{.5em} \email \par
            \vspace{12pt}
            \iconoDireccion \hspace{.5em} Mislata, Valencia, Spain \par
            \vspace{12pt}
            \iconoLinkedin \hspace{.5em} \linkedinpage \par
            \vspace{12pt}
            \iconoGitHub \hspace{.5em} \githubpage \hspace{1em} \iconoGitLab \hspace{.5em} \gitlabpage
        \end{large}
    \end{minipage}
    \begin{minipage}[r]{.25\textwidth}
        \centering
        \href{\linkToSelf}{\linkToSelfShort}
        {
            \hypersetup{
                urlcolor=Black
            }
            \qrcode[padding]{\linkToSelf}
        }
    \end{minipage}
    \begin{center}
    Alejandro Gómez is a \textbf{Software engineer},\\currently \textbf{working} at the \textbf{\href{\ujiUrl}{\ujiFull}}.
    
    %\\currently \textbf{working} at the \textbf{\href{\sabienUrl}{\itacaSabien}} research group\\ from the \textbf{\href{\upvUrl}{\upvFull}}.
    \end{center}
}

% Other info
\newcommand{\upvFull}{{\upvName} ({\upv})}
\newcommand{\upvName}{Polytechnic University of Valencia}
\newcommand{\upv}{UPV}
\newcommand{\upvUrl}{https://www.upv.es/en}

\newcommand{\sabien}{SABIEN}
\newcommand{\itacaSabien}{ITACA-SABIEN}
\newcommand{\sabienUrl}{http://www.sabien.upv.es/en/}

\newcommand{\ujiFull}{{\ujiName} ({\uji})}
\newcommand{\ujiName}{Universitat Jaume I}
\newcommand{\uji}{UJI}
\newcommand{\ujiUrl}{https://www.uji.es/upo/rest/publicacion/idioma/en?urlRedirect=https://www.uji.es/&url=/}
\newcommand{\ujiGroup}{Knowledge Engineering}
\newcommand{\ujiGroupUrl}{https://www.uji.es/upo/rest/publicacion/idioma/en?urlRedirect=https://www.uji.es/serveis/ocit/base/grupsinvestigacio/detall/&url=/serveis/ocit/base/grupsinvestigacio/detall&codi=158}

\newcommand{\fce}{\emph{First Certificate in English - Grade A - Cambridge English Level 2}}

\begin{document}

\title{Curriculum Vitae}
\author{Alejandro Gómez Noé}

\maketitle

\contactinfo

\vspace{8pt}

\rule{.9\textwidth}{.4pt}

\section*{Knowledge}

\vspace{.8em}

\begin{adjustwidth}{-1.5cm}{}
    \begin{tikzpicture}

        % Define the styles for the circles
        \tikzset{
            skill/.style={
                    align=center,
                    inner sep=0pt,
                    font=\bfseries
                },
            skill-circle/.style={
                    circle,
                    draw,
                    text width=2cm,
                    align=center,
                    inner sep=0pt,
                    font=\bfseries
                },
            category/.style={
                    circle,
                    draw,
                    text width=6cm,
                    minimum height=3cm,
                    align=center,
                    text depth=6cm,
                    font=\bfseries
                },
            category-small/.style={
                    circle,
                    draw,
                    text width=4cm,
                    minimum height=3cm,
                    align=center,
                    text depth=3cm,
                    font=\bfseries,
                },
            category-smaller/.style={
                    circle,
                    draw,
                    text width=3.0583cm,
                    align=center,
                    text depth=3cm,
                    font=\bfseries,
                },
            subcategory/.style={
                    circle,
                    draw,
                    dashed,
                    text width=2.5cm,
                    minimum height=3cm,
                    align=center,
                    text depth=3.5cm,
                    font=\bfseries
                }
        }

        % Draw the categories
        \node[category-smaller] (db) at (1.1,-2.5) {Databases};
        \node[category-smaller, text width=2.5cm, text depth=3cm] (vcs) at (6.8,-0.25) {Version control};
        \node[category-small] (tools) at (5.9,-4.5) {Tools};
        \node[subcategory] (dotnet) at (-0.3,2.3) {\includesvg[scale=2]{dotnet.svg}};
        \node[category] (web) at (-5,1) {Web development};
        \node[category-smaller, align=center, text depth=1.5cm] (mobile) at (5,3.5) {Mobile};
        \node[category-small, align=right, text depth=0] (desktop) at (2,2) {\colorbox{white}{Desktop}};

        % Draw the skills within the categories
        \node[skill] (html) at (-8.3,2.5) {{\fontsize{30pt}{30pt}\faHtml5}\\\textbf{HTML}};
        \node[skill] (css) at (-6.5,2.5) {{\fontsize{30pt}{30pt}\faCss3*}\\\textbf{CSS}};
        \node[skill] (js) at (-4.6,2.45) {{\fontsize{30pt}{30pt}\faJs}\\\textbf{JavaScript}};
        \node[skill] (php) at (-6,0.35) {\fontsize{40pt}{40pt}\faPhp};
        \node[skill,text width=2.5cm] (wordpress) at (-8.2,0.25) {{\fontsize{30pt}{30pt}\faWordpress}\\\textbf{WordPress}};
        \node[skill] (elementor) at (-6.5,-1.5) {{\fontsize{30pt}{30pt}\faElementor}\\\textbf{Elementor}};
        \node[skill] (ts) at (-3.8,0.25) {{\fontsize{30pt}{30pt}\includesvg[scale=1.3]{typescript.svg}}\\\textbf{TypeScript}};
        \node[skill] (node) at (-4.3,-2.2) {{\fontsize{30pt}{30pt}\faIcon{node-js}}\\\textbf{Node.js}};
        \node[skill] (docker) at (-2.5,-1.5) {{\fontsize{30pt}{30pt}\faDocker}\\\textbf{Docker}};
        \node[skill, text width=2.1cm] (csharp) at (-0.4,3.1) {{\fontsize{30pt}{30pt}\includesvg[scale=1.3]{csharp.svg}}};
        \node[skill-circle, text width=1.5cm] (wpf) at (1.3,3) {WPF};
        \node[skill] (unity) at (0.9,1.3) {{\fontsize{30pt}{30pt}\faUnity}\\\textbf{Unity}};
        \node[skill, text width=2.1cm] (blazor) at (-1.8,2) {{\fontsize{30pt}{30pt}\includesvg[scale=1.3]{blazor.svg}}\\\textbf{Blazor}};
        \node[skill] (java) at (2.6,5) {{\fontsize{30pt}{30pt}\faJava}\\Java};
        \node[skill] (flutter) at (5.2,3) {{\fontsize{30pt}{30pt}\includesvg[scale=1.1]{flutter.svg}}\\Flutter};
        \node[skill] (sql) at (0,-2.5) {{\fontsize{30pt}{30pt}\faDatabase}\\SQL};
        \node[skill] (firebase) at (1.9,-2.5) {{\fontsize{30pt}{30pt}\includesvg[scale=1.2]{firebase.svg}}\\Firebase};
        \node[skill] (alexa) at (-2,-3.5) {{\fontsize{30pt}{30pt}\includesvg[scale=1.2]{alexa.svg}}\\Alexa};
        \node[skill] (linux) at (3.9,-0.6) {{\fontsize{30pt}{30pt}\faLinux}\\Linux};
        \node[skill] (git) at (5.7,-0.5) {{\fontsize{30pt}{30pt}\faGit}};
        \node[skill] (github) at (7.5,-0.5) {{\fontsize{30pt}{30pt}\faGithub}\\GitHub};
        \node[skill] (vscode) at (4.6,-4.2) {{\fontsize{30pt}{30pt}\includesvg[scale=1.1]{visual-studio.svg}}\\Visual Studio};
        \node[skill] (vscode) at (7.3,-4.2) {{\fontsize{30pt}{30pt}\includesvg[scale=1.1]{vscode.svg}}\\VS Code};
        \node[skill] (trello) at (6,-6) {{\fontsize{30pt}{30pt}\faTrello}\\Trello};
        \node[skill] (latex) at (2.25,-5.5) {\Large\LaTeX};
    \end{tikzpicture}
\end{adjustwidth}

\pagebreak

\section*{Experience}

\subsection*{May 2024 - February 2025 | Research Support Staff | {\ujiFull}}

As \textbf{research support staff} in the \emph{\textbf{\href{\ujiGroupUrl}{\ujiGroup}}} group at the \emph{\href{\ujiUrl}{\ujiFull}}, my work consists of \textbf{research} and \textbf{application development} within the framework of the \textbf{research project \emph{MINEGUIDE}}.

\subsubsection*{MINEGUIDE Project}

The full name of the project is: ``\emph{Integrated development and exploitation of process mining models and clinical guideline models in support of the Learning Health System (MINEGUIDE)}'', and it is a collaborative project between 3 Spanish universities:
\begin{itemize}
    \item \textbf{\href{\ujiUrl}{\uji}} - {\ujiName} (Castellón) (\emph{coordinator})
    \item \textbf{\href{\upvUrl}{\upv}} - {\upvName} (Valencia)
    \item \textbf{\href{https://www.um.es/}{UMU}} - University of Murcia (Murcia) 
\end{itemize}

whose objective is the development of a \textbf{decision support system} in the field of \textbf{healthcare}.\\

The project is based on \textbf{process mining} and \textbf{clinical guidelines}, and focuses on improving \textbf{care quality} and \textbf{efficiency} of healthcare processes by finding \textbf{relationships} between \textbf{real processes} (obtained from data) and \textbf{ideal processes} (defined in clinical guidelines).\\

My main work has been to conduct a \textbf{case analysis}, applying \textbf{process mining} techniques to a set of hospital data related to \textbf{COPD} (\emph{Chronic Obstructive Pulmonary Disease}), with the aim of finding if there are \textbf{relationships} or \textbf{deviations} between the real and ideal processes, publishing the results in a scientific article.

\subsection*{March 2022 - April 2024 | Mid-level software engineer | {\itacaSabien} (\upv)}

As a \textbf{mid-level software engineer} at the \emph{\textbf{\href{\sabienUrl}{SABIEN}} (Technological innovations for Health and Wellbeing)} group from the \emph{Institute of Applied Information Technologies and Advanced Communications (\textbf{\href{http://www.itaca.upv.es/}{ITACA}})}, which is part of the \emph{{\upvName} (\textbf{\href{\upvUrl}{\upv}})}, my work consists on \textbf{developing applications} and providing \textbf{technical support} in various ways in the context of several \textbf{research projects}.

\subsubsection*{Projects I have participated in}

\vspace{1em}

\begin{itemize}
    \item {\large\textbf{\href{https://moveit.webs.upv.es/}{MOVE-IT}}} (2023 - 2024)
          \begin{itemize}
              \item Training program for improving physical exercise of people with intellectual disabilities through \textit{exergames} and technology
              \item \textbf{European ERASMUS+ project} in collaboration with:
                    \begin{itemize}
                        \item \textbf{\href{https://www.ivass.gva.es/}{IVASS}} - \textit{Instituto Valenciano de Servicios Sociales} (Spain)
                        \item \textbf{\href{https://www.cercioeiras.pt/pt}{CERCIOEIRAS}} - \textit{Cooperativa de Educação e Reabilitação de Cidadãos com Incapacidade} (Portugal)
                        \item \textbf{\href{https://en.uit.no/}{UiT}} - Tromsø University (Norway)
                        \item \textbf{\href{https://www.ospedalemotta.it/it/}{ORAS}} - \textit{Ospedale riabilitativo di Motta di Livenza} - Hospital (Italy)
                    \end{itemize}
              \item I collaborated with \textbf{UiT} in the development of the apps for the project (\textit{AGA} and \textit{Sorterius}), built with \textbf{Unity}
              \item I implemented a website with an API for data collection using \textbf{Blazor}, \textbf{ASP.NET Core Identity} and \textbf{EFCore}, which I then hosted in a \textit{\upv} server
              \item I integrated the \textit{Sorterius} app with said API for user management and data collection during the pilots for later study
              \item I managed the server and worked alongside the participant centers during the pilots to ensure the apps were working properly
          \end{itemize}
    \item {\large\textbf{\href{https://pm4health.com/}{PM4H}}} (2022 - present)
          \begin{itemize}
              \item Usage of \textbf{\emph{Process Mining}} techniques for improving efficiency on the management of information in the health sector
              \item I developed several features for the ``\textbf{\emph{PMApp}}'' \textbf{desktop application}, which runs \textbf{\emph{process mining}} algorithms for treating and visualizing different kinds of data
              \item Said app is developed in \textbf{C\#}, and uses \textbf{WPF} for the user interface
              \item I have improved the visualization system for tables and histograms, I have added support for using \emph{proxies}\dots
          \end{itemize}
    \item {\large\textbf{\href{https://lifechamps.eu/}{LIFECHAMPS}}} (2023)
          \begin{itemize}
            \item \textbf{European project} for improving the quality of life of cancer patients, consisting of a consortium of 15 partners from 10 countries, led by the \href{https://www.auth.gr/en/}{Aristotle University of Thessaloniki}
            \item I closely collaborated with the \textbf{\href{https://www.iislafe.es/es/}{IIS La Fe}} and the company \textbf{\href{https://www.mysphera.com/es/}{MySphera}}, \textbf{configuring} \textbf{Raspberry Pi} devices and performing a total of \textbf{50 installations} in patients' homes in the province of Valencia between March and July 2023 for a multinational pilot project
          \end{itemize}
    \item {\large\textbf{\href{http://www.sabien.upv.es/project/dial/}{DIAL}}} (2022 - 2023)
          \begin{itemize}
              \item \textbf{Voice assistant} for the detection and addressing of Unwanted Loneliness in the elderly, based on the \emph{open source} \textbf{\href{https://mycroft.ai/}{Mycroft}} system
              \item My contribution included training a \textbf{\emph{machine learning}} model so the phrase ``\emph{Hola dial}'' can be used as a wake word for the assistant
              \item On top of that, I have \textbf{configured} over 20 \textbf{Raspberry Pi} devices so they function as voice assistants by means of the DIAL system
          \end{itemize}
    \item {\large\textbf{\href{https://orriolsarrandeterra.com/}{Orriols Arran de Terra}}} (2022)
          \begin{itemize}
              \item \textbf{Website} intended for broadcasting news and local activities from the \emph{Els Orriols} neighborhood in Valencia
              \item I developed said website using \textbf{WordPress} and \textbf{Elementor}
              \item In order to create some custom features, I built \textbf{plugins} with \textbf{PHP}
              \item I also used \textbf{HTML}, \textbf{CSS} y \textbf{JavaScript}
          \end{itemize}
\end{itemize}

\section*{Education}

\subsection*{University}

\fbox{
    \large
    \begin{minipage}{.6\textwidth}
        Computer Science degree

        Mention in Software Engineering

            {\upvFull}, 2021
    \end{minipage}
}

\subsubsection*{Projects for subjects}

\begin{itemize}
    \item \textbf{Al Loro} (\href{https://github.com/algono/FeedTheParrot-RSS}{Repositorio}) (\href{http://hdl.handle.net/10251/174256}{Memoria}):
          \begin{itemize}
              \item I implemented by myself an Amazon \textbf{Alexa} \textbf{skill} for my \textbf{final degree project}, using \textbf{Node.js} and \textbf{TypeScript}.
              \item I integrated said skill with a \textbf{database} hosted in \textbf{Firebase}, on top of creating a mobile app using \textbf{Flutter} to manage user preferences
              \item I designed an authentication system using several \textbf{AWS} services; such as \textbf{Lambda}, \textbf{DynamoDB} or \textbf{API Gateway}
          \end{itemize}
    \item \textbf{Frozen Out} (\href{https://github.com/Freezer-Games/Frozen-Out}{Repositorio}):
          \begin{itemize}
              \item I was part of the development team for a video game made in \textbf{Unity} with \textbf{C\#} as a project for the ``\emph{IPV}'' subject (2019)
              \item I designed a \textbf{dialog system} with support for using formats such as bold, italics, and different colors by working on top of the \href{https://yarnspinner.dev/}{\textbf{\emph{YarnSpinner}}} library
              \item This project was part of the \href{https://es-es.facebook.com/etsinf/videos/feria-de-proyectos-de-estudiantes-2019/1921312964681641/}{\textbf{Project fair} (2019)} organized by the \emph{ETSINF} (computer science school at the \emph{\upv})
              \item The project carried on without me after the final submission for the subject (January 2020). In February of 2021, \emph{Frozen Out} won the \textbf{PlayStation Commitment Special Award} (\href{https://www.inf.upv.es/www/etsinf/es/premio-especial-compromiso-playstation-para-el-videojuego-frozen-out-creado-por-estudiantes-de-la-etsinf-y-la-facultat-de-bb-aa/}{article in Spanish})
          \end{itemize}
\end{itemize}

\section*{Languages}

\begin{itemize}
    \item \textbf{Spanish}, native
    \item \textbf{English}, \textbf{C1} level (\fce)
    \item Catalan (Valencia), C1 level
\end{itemize}

\section*{Activities}

\subsection*{Mentor - Technovation Challenge}

I participated as a volunteer mentor for the \href{https://technovationchallenge.org/}{Technovation Challenge} contest organized by Iridescent in its 2019 edition, in collaboration with \href{https://americanspacev.upv.es/}{American Space}, an association from the \emph{\upvFull}.

\section*{Others}

\begin{itemize}
    \item Driving license (class B)
\end{itemize}

\end{document}